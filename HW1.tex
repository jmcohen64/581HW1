\documentclass[11pt,letterpaper]{article}
\usepackage[top=.5in,textheight=9in]{geometry}
\usepackage{amsmath, amsthm, amssymb}
\usepackage{enumerate}
\usepackage{xfrac}


% Everything after a % sign is commented out.
% This is sometimes useful to write notes to yourself
% or to add spacing in the tex file so that it is easier
% to read.


% Some useful `macros'
% % % Feel free to define your own!
\newcommand{\C}{\mathbb{C}}
\newcommand{\N}{\mathbb{N}}
\newcommand{\R}{\mathbb{R}}
\newcommand{\Z}{\mathbb{Z}}

\newcommand{\cS}{\mathcal S}


% Here is a pretty way to write down the problem
\newtheorem{innerprob}{Problem}
\newenvironment{prob}[1]
{\renewcommand\theinnerprob{#1}\innerprob}
{\endinnerprob}
% Here is a pretty way to wrote down the solution
\newenvironment{solution}
{\renewcommand\qedsymbol{}\begin{proof}[Solution]}
	{\end{proof}\bigskip}

\setlength\parindent{0cm}
\setlength\parskip{5pt plus 1pt minus 1pt}


\title{Assignment \#1\\Math 581A}
\author{
	John Cohen
}
\date{September 10th 2024}









\begin{document}
	
	\maketitle
	
\begin{solution}[Exercise 2.1]
	Let $P_n(x) = a_0 + a_1x + \dotsc + a_nx^n$ be a polynomial of order $n$ with each $a_n \in \R$. If $\underline z \in \C$ is a root of $P_n(x)$, then $$P_n(x = \underline z) = a_0 + a_1\underline z + \dotsc + a_n\underline z^n = 0.$$ Now consider when $x = \underline z^*$: $$P_n(x = \underline z^*) = a_0 + a_1\underline z^* + \dotsc + a_n{\underline z^*}^n.$$ By Theorem 2.1.4, 
	\[\begin{split}
		P_n(x = \underline z^*) &= (P_n(x = \underline z))^*\\
		&= 0^*\\
		&= 0.
	\end{split}\]
	Therefore, $\underline z^*$ is also a root of $P_n(x)$. 
\end{solution}

\begin{solution}[Exercise 2.2]
	Let $f(z) = e^z$. With the substitution $z = x + iy$,
	\[\begin{split}
		f(z) &= e^{x+iy}\\
		&=e^xe^{iy}\\
		&=e^x(\cos(y)+i\sin(y))\\
		&= e^x\cos(y) +ie^x\sin(y).
	\end{split}\]
	Let $z(x) = x + i\hat y$ for some fixed $\hat y$. As $x \to \infty$, $f(z(x)) \to \infty$ and as $x \to -\infty$, $f(z(x)) \to 0$. Now let $z(y) = \hat x + iy$ for some fixed $\hat x$. Since $f(z(y))$ is periodic in $y$, $y\to \infty$ gives a circle with radius $e^{\hat x}$. Thus, if $z = re^{i\theta}$ describes a line through the origin, for a fixed $\theta$, we get a spiral.
	
	Let $f(z) = \sin(z)$ using the exponential formula for $\sin$ and letting $z = x+iy$ gives
	\[\begin{split}
		f(z) &= \frac{e^{iz}-e^{-iz}}{2i}\\
		&= \frac{e^{i(x+iy)}-e^{-i(x+iy)}}{2i}\\
		&= \frac{e^{ix}e^{-y} -e^{-ix}e^{y}}{2i}\\
		&= \frac{e^{-y}(\cos(x)+i\sin(x))-e^y(\cos(-x)+i\sin(-x))}{2i}\\
		&= \frac{e^{-y}(\cos(x)+i\sin(x))-e^y(\cos(x)-i\sin(x))}{2i}\\
		&= \frac{\cos(x)}{2i}(e^{-y}-e^y)+\frac{i\sin(x)}{2i}(e^{-y}+e^y)\\
		&= \frac{-i\cos(x)}{2}(e^{y}-e^{-y})+\frac{\sin(x)}{2}(e^{-y}+e^y)\\
		& = \sin(x)\cosh(y) + i\cos(x)\sinh(y)
	\end{split}\]
	\end{solution}	

\end{document}